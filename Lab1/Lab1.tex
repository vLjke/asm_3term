\documentclass[a4paper,12pt]{article} 

%%% Работа с русским языком
\usepackage{cmap}					% поиск в PDF
\usepackage{mathtext} 				% русские буквы в фомулах
\usepackage[T2A]{fontenc}			% кодировка
\usepackage[utf8]{inputenc}			% кодировка исходного текста
\usepackage[english,russian]{babel}	% локализация и переносы

%%% Дополнительная работа с математикой
\usepackage{amsmath,amsfonts,amssymb,amsthm,mathtools, gensymb} % AMS
\usepackage{icomma} % "Умная" запятая: $0,2$ --- число, $0, 2$ --- перечисление

%%Таблица
\usepackage[table,xcdraw]{xcolor}
\usepackage{caption}
\usepackage{subcaption}
\usepackage{floatrow}
\floatsetup[table]{capposition=top}
\floatsetup[wrapfigure]{capposition=bottom}


%% Номера формул
\mathtoolsset{showonlyrefs=true} % Показывать номера только у тех формул, на которые есть \eqref{} в тексте.

%% Шрифты
\usepackage{euscript}	 % Шрифт Евклид
\usepackage{mathrsfs} % Красивый матшрифт

%% Свои команды
\DeclareMathOperator{\sgn}{\mathop{sgn}}

%% Перенос знаков в формулах (по Львовскому)
\newcommand*{\hm}[1]{#1\nobreak\discretionary{}
{\hbox{$\mathsurround=0pt #1$}}{}}

%% Стиль страницы
\usepackage{fancyhdr}

%% Для рисунков
\usepackage{graphicx}
\usepackage[export]{adjustbox}
\usepackage{float}
\usepackage{ragged2e}
\usepackage{wrapfig}

%Отступы и поля 
\textwidth=20cm
\oddsidemargin=-2cm
\topmargin=-2cm
\textheight=25cm

\pagestyle{fancy}
\begin{document}
\begin{titlepage}
\begin{center}
%\vspace*{1cm}
\large{\small ФЕДЕРАЛЬНОЕ ГОСУДАРСТВЕННОЕ АВТОНОМНОЕ ОБРАЗОВАТЕЛЬНОЕ\\ УЧРЕЖДЕНИЕ ВЫСШЕГО ОБРАЗОВАНИЯ \\ МОСКОВСКИЙ ФИЗИКО-ТЕХНИЧЕСКИЙ ИНСТИТУТ\\ (НАЦИОНАЛЬНЫЙ ИССЛЕДОВАТЕЛЬСКИЙ УНИВЕРСИТЕТ)\\ ФАКУЛЬТЕТ АЭРОКОСМИЧЕСКИХ ТЕХНОЛОГИЙ}
\vfill
\line(1,0){430}\\[1mm]
\huge{Лабораторная 1}\\
\huge\textbf{База + фундамент}\\
\line(1,0){430}\\[1mm]
\vfill
\begin{flushright}
\normalsize{Рогозин Владимир}\\
\normalsize{\textbf{Группа Б03-106}}\\
\end{flushright}
\end{center}
\end{titlepage}
\fancyhead[L] {Лабораторная 1}

\textbf{Пункты 3, 4, 5:}

Сначала посмотрим как выглядит простейшая программа, которая выводит строку "Hello world" на языке ассемблера, сравним ассемблерный листинг простейшей программы на языке С с точно такой же программой на языке С++.
\begin{figure}[H]
    \subfloat["Hello world" на С]{
    \begin{minipage}[t][1\width]{0.395\textwidth}
        \centering
        \includegraphics[width= \textwidth]{HelloWorld-C.png}
    \end{minipage}}
    \hfill
    \subfloat["Hello world" на С++]{
    \begin{minipage}[t][1\width]{0.583\textwidth}
        \centering
        \includegraphics[width= \textwidth]{HelloWorld-C++.png}
    \end{minipage}}
    \label{fig:Hello world C++ and C}
\end{figure}

Не трудно заметить, что имеются небольшие различия -- на C++ добавилась дополнительная информация, программа стала объёмнее. Однако прослеживается схожесть во многих командах, к примеру строка "Hello world" объявляется в обоих языках одинаково. Ниже представлен код программы, которая создаёт три глобальных целочисленных переменных, двум из них присваиваются значения, а третьей присваивается произведение значений первых двух. Опять же, видны различия в объявлении переменных, но те части кода, где происходит присваивание и умножение абсолютно идентичны на С и С++.  

\begin{figure}[H]
    \subfloat[Умножение двух целых чисел на С]{
    \begin{minipage}[t][1\width]{0.5\textwidth}
        \centering
        \includegraphics[width= \textwidth]{umnojenieC.png}
    \end{minipage}}
    \hfill
    \subfloat[Умножение двух целых чисел на С++]{
    \begin{minipage}[t][1\width]{0.4745\textwidth}
        \centering
        \includegraphics[width= \textwidth]{umnojenieCpp.png}
    \end{minipage}}
    \label{fig:umnojenie C++ and C}
\end{figure}

\newpage
Далее, добавим вывод на экран(чтобы проверять корректность работы программы), сложим два числа, запишем результат в третье, выведем его на экран. После этого попробуем изменить программу непосредственно через ассемблерный листинг, сравним результаты работы обеих программ. Ниже приведены скриншоты различий программ и вывода в консоль до/после.

\begin{figure}[H]
    \subfloat[Сложение 4 с 5]{
    \begin{minipage}[t][1\width]{0.478\textwidth}
        \centering
        \includegraphics[width= \textwidth]{slojenieCBefore.png}
    \end{minipage}}
    \hfill
    \subfloat[Сложение 44 с 5]{
    \begin{minipage}[t][1\width]{0.481\textwidth}
        \centering
        \includegraphics[width= \textwidth]{slojenieCAfter.png}
    \end{minipage}}
    \label{fig:slojenie C++ and C}
\end{figure}

\begin{figure}[H]
    \subfloat[Результат сложения 4 с 5]{
    \begin{minipage}[t][1\width]{0.48\textwidth}
        \centering
        \includegraphics[width= \textwidth]{resultSlojeniaCBefore.png}
    \end{minipage}}
    \subfloat[Результат сложения 44 с 5]{
    \begin{minipage}[t][1\width]{0.48\textwidth}
        \centering
        \includegraphics[width= \textwidth]{resultSlojeniaCAfter.png}
    \end{minipage}}
    \label{fig:resul'tat slojenia C++ and C}
    \vspace{-8cm}
\end{figure}

Таким образом, изменив в ассемблерном листинге значение, присваиваемое одному из чисел, поменяли значение третьей переменной и результат работы программы. Исходный файл operations.c при этом не изменился.
\newpage

\textbf{Пункт 6:}

Теперь выясним какие команды отвечают за операции сложения, вычитания, присваивания. 

\begin{figure}[H]
    \subfloat[Команда addl складывает числа]{
    \begin{minipage}[t][1\width]{0.48\textwidth}
        \centering
        \includegraphics[width= \textwidth]{slojenie.png}
    \end{minipage}}
    \subfloat[Команда subl вычитает числа]{
    \begin{minipage}[t][1\width]{0.4758\textwidth}
        \centering
        \includegraphics[width= \textwidth]{vychitanie.png}
    \end{minipage}}
    \label{fig:resul'tat slojenia C++ and C}
\end{figure}
Не трудно увидеть, что за сложение отвечает команда \textit{\textbf{addl}}, за вычитание -- \textit{\textbf{subl}}. Присваиваются значения с помощью команды \textit{\textbf{movl}}.

\textbf{Пункт 7:}

Создадим несколько глобальных переменных различного типа, посмотрим как они объявляются, проделаем некоторые операции с ними. Видно, что глобальные переменные объявляются в самом верху, рядом с названием указан размер переменной в байтах и ещё какое-то число. В данной простейшей программе присваиваются значения глобальным переменным, затем эти значения изменяются: переменной типа \textit{char} присваивается другая буква, \textit{int} складывается с \textit{long} и результат записывается во \textit{float}. 

\begin{figure}\label{fig: global vars}
    \centering
    \includegraphics[width = 0.6\textwidth]{7chapter.png}
    \caption*{Четыре глобальные переменные различных типов и действия с ними}
\end{figure}
\newpage

\textbf{Пункт 8:}

Из предыдущей программы видно, что при выполнении арифметических действий значения различных по типу переменных копируются в различные регистры в зависимости от размера памяти, занимаемого переменной. Изначально переменная типа \textit{int} кладётся в регистр \textit{eax}, размер которого 32 бита, что совпадает с размером \textit{int}'а, а переменная типа \textit{long} в регистр \textit{rax}, который содержит в себе 64 бита(как и размер переменной типа \textit{long}). Перед тем как сложить \textit{int} c \textit{long} переменная типа \textit{int} копируется в 64-битный регистр \textit{rdx}, затем складывается с \textit{long}'ом, потом результат сложения записывается в \textit{rax}, после этого это число записывается в знаяение переменной типа \textit{float}. Также стоит отметить, что результат арифметических операций записывается во второй регистр. Таким образом, обращение к регистру происходит через знак \%, а именно \%eax, также, если в регистре лежит адрес переменной, то к значению переменной можно обратиться сразу, добавив скобочки вот так вот (\%eax)(эту инфу нагуглил). К тому же, для хранения различных данных, в зависимости от размера, используются различные регистры.

\textbf{Пункт 9:}

В этом пункте посмотрим на умножение и деление беззнаковых и знаковых чисел различных размеров. За знаковое деление и умножение отвечают \textit{\textbf{imul}} и \textit{\textbf{idiv}} соответственно. Внизу представлена программа с двумя числами типа \textit{int}, в которой сначала во вторую переменную записывается результат их перемножения, а потом результат их деления: 
\begin{figure}[H]\label{fig: imull idivl}
    \centering
    \includegraphics[width = 0.4\textwidth]{imullidivl.png}
    \caption*{Умножение и деление целых знаковых чисел типа \textit{int}}
\end{figure}
\newpage

Видим, что у каждой из команд появилась буква $l$ в конце, которая указывает на размер операндов. Обе переменных 32-х битные, поэтому используются регистры \textit{edx}, \textit{eax} и \textit{ecx}. У \textit{\textbf{imul}} два аргумента, результат записывается во второй регистр(\textit{eax} в данном случае). У команды \textit{\textbf{idiv}} всего один аргумент и, судя по действиям перед и после деления, команда \textit{\textbf{idiv}} берёт делимое из регистра \textit{eax} по умолчанию, а делитель -- из единственного аргумента, и сохраняет результат в \textit{eax}. Теперь сделаем те же арифметические операции с беззнаковыми числами:
\begin{figure}[H]\label{fig: imull divl}
    \centering
    \includegraphics[width = 0.4\textwidth]{imulldivl.png}
    \caption*{Умножение и деление целых беззнаковых чисел типа \textit{int}}
\end{figure}

Можем видеть, что вместо команды \textit{\textbf{idivl}} появилась \textit{\textbf{divl}}, которая отвечает за деление беззнаковых чисел, при этом умножаются числа по-прежнему командой \textit{\textbf{imull}}, результат умножения все так же записывается во второй аргумент(регистр), операция \textit{\textbf{divl}} по-прежнему берет делимое из регистра \textit{eax}, и туда же записывает результат деления. Делитель берётся из регистра \textit{ecx}.

Теперь в качестве типов переменных возьмём \textit{long long} и \textit{unsigned long long}. Видно, что поменялись только последние буквы в командах \textit{\textbf{imulq}}, \textit{\textbf{idivq}} и \textit{\textbf{divq}}. Также, так как изменился размер переменной, то изменились и регистры, в которых хранятся результаты операций и значения самих переменных. Команду \textit{\textbf{mul}} так и не удалось вытащить, но работает она схожим с \textit{\textbf{div}} и \textit{\textbf{idiv}}\textbf{} образом(в том смысле, что принимает один аргумент, а второй сомножитель заведомо находится в определенном регистре).

\begin{figure}[H]
    \subfloat[Умножение и деление целых знаковых чисел \\ типа \textit{long long}]{
    \begin{minipage}[t][1\width]{0.48\textwidth}
        \centering
        \includegraphics[width= \textwidth]{imulqidivq.png}
    \end{minipage}}
    \subfloat[Умножение и деление целых беззнаковых чисел \\ типа \textit{long long}]{
    \begin{minipage}[t][1\width]{0.484\textwidth}
        \centering
        \includegraphics[width= \textwidth]{imulqdivq.png}
    \end{minipage}}
    \label{fig:resul'tat slojenia C++ and C}
\end{figure}

\end{document}
